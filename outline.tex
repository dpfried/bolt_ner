\documentclass{article}
\usepackage[margin=1in]{geometry}
\title{Bootstrapped Noun Phrase Resolution with Naive Bayes Classifiers}

\begin{document}
\maketitle

\section{Experiment Setup}
\subsection{Scenes}
\begin{itemize}
\item Incremental introduction of objects 
\item Generated scene from a schematic, with information about color, shape, location, and size of objects.
\end{itemize}

\subsection{Resposnes}
\begin{itemize}
\item Average of 29 responses per scene.

\item Widely varying quality of responses (show examples)
\end{itemize}

\section{Building Classifiers}
\begin{itemize}
\item Bag of words classification using NBCs, maximum a posteriori, and add-one smoothing.

\item Two different approaches:
\subsection{Classifying Distinct Objects}
\begin{itemize}
    \item Building classifiers for objects, within a scene, based on knowledge of which object is new
    \item Noun-phrase-in-subject isolation as a substitute for semiotics. Using TRIPS and parse tree pruning
\end{itemize}
\subsection{Classifying Features}
\begin{itemize}
    \item Building classifiers for features (shape and color), based on knowledge of the objects' features
    \item Can train on all scenes, not just scenes with a given sequence
\end{itemize}
\end{itemize}

\section{Evaluation}
\begin{itemize}
\item Building a goldstandard for proofing (not used to train classifier in any way).

\item Results on object classification. Present k-fold cross validation overall (87\%
object recognition accuracy), and with learning curves.

\item Results on feature classification. Present k-fold cross validation overall (86\%
for color, 90\% for shape), and with learning curves.

\item When a larger training set is available (e.g. when training features, not
objects), have had good results even with more inclusive feature extraction.
Training on all innermost noun phrases (79\% for color, 87\% for shape), and even all words, not using the parse tree at
all (81\% color, 86\% for shape). These are k-fold cross validation scores, separating training from test data.

\item In all iterations, higher accuracy on shape recognition than color
recognition. Possibly because there were fewer shape classes (3 shape classes,
5 color classes), or that it's more difficult for humans to tell colors apart.
(Cyan had the lowest recognition accuracy, confusion with blue?)
\end{itemize} 

\section{Future directions}

\end{document}
